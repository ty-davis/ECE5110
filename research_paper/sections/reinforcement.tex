% !TeX root = ..\research_paper.tex

Reinforcement learning (RL, or DRL for deep reinforcement
learning) is a form of machine learning where an agent
will perform a task (such as floorplanning or routing)
and then the solution will be evaluated and given either
a punishment or a reward. The agent then ``learns''
from that reward. The solutions that are given rewards
will be replicated and iterated on, while those that
are not will be rejected in future iterations. Hundreds
of thousands of examples will be used in order for the
neural network to effectively learn and perform in a
way that fits design needs. A group from the Google
Chip Implementation showed their results and methods
in a 2020 paper. They mentioned that the reward system
for their DRL was based on ``wirelength, because it
is not only much cheaper to evaluate, but also correlates
with power and performance (timing)'' (\cite{DBLP:journals/corr/abs-2004-10746}).

The paper outlined performance improvements that resulted
from using RL instead of simulated annealing processes. On
average, wirelength and congestion decreased, meaning that
overall power consumption and performance would be improved.
