% !TeX root = ..\research_paper.tex

The technique of ``simulated annealing'', much like
the A$^*$ algorithm, has many uses beyond VLSI design,
and being such a versatile tool can actually be used in
multiple places in VLSI design, so it should be mentioned
here.

Simulated annealing can be used in floorplanning, cell placement,
and even detail routing, as it's just a technique that wraps
around an algorithm, rather than replacing it. As a process,
it avoids local minima by allowing iterations on the starting
conditions of an algorithm to vary widely at the start, and 
more narrowly as computation continues. It gets its name from
a processes in metallurgy known as ``annealing'' where materials
are heated and cooled to reduce defects.

On the high-level, here is how it works. Starting at
some initial condition, the algorithm iteratively explores
the solution space by making small random changes. After
a given iteration, the solution is either accepted or
rejected. If the conditions improve, then it is accepted,
but if the conditions worsen the solution isn't necessarily
rejected. The earlier iterations have a higher ``temperature''
value which gradually decreases as the iterations go
on. The higher the temperature value, the worse that
a potential solution can be while still being selected.
By allowing potential worse solutions to be accepted
at the start, simulated annealing increases the chances
of escaping local extrema, in an effort to find the
global minimum or maximum of the function. 

Simulated annealing finds such a useful place in VLSI
design because many of the problems are NP-hard problems,
where the compute time to find an optimal solution is
unrealistic. An example using simulated annealing in
VLSI could be in floorplanning, where exhaustively selecting
the locations of hundreds of thousands of cells would
be computationally impossible. So, the algorithm would
suppose starting from a base condition where all of
the cells are placed randomly on a substrate. After
analyzing the possible solutions, moving some of the
cells randomly (or heuristically) the solution is then
analyzed. If the condition improves it is selected.
But, based on the process of simulated annealing, if
the temperature value is still high (e.g. it is an early
iteration), a worse condition may be selected. Possible
worse floorplanning will be chosen an iterated on, as
this will reduce the chance of falling into a local
extrema.
