% !TeX root = ..\research_paper.tex

As of the early 2000's, ``most routing algorithms use[d]
a Manhattan geometry with horizontal and vertical traces''
(\cite{1360580}). Such designs prevailed because of
their simplicity, but the increase in wire length (27\%
longer on average than the Euclidean distance) of the
interconnections resulted in significantly slower circuits.
The concept of ``non-Manhattan'' routing, therefore,
is any routing that doesn't strictly adhere to a grid.

Non-Manhattan layout designs can be applied not only
to VLSI design projects, but also PCB and other circuit
layouts. The study cited above showed the use of two
different Non-Manhattan layout algorithms that routed
wires on horizontal, vertical, and \ang{45} diagonal
lines, but other non-Manhattan layout paradigms exist
which operate on other bases. While the results from
that study can only be applied to the specific layout
algorithms tested, it is safe to say that Non-Manhattan
layouts should always be a consideration when trying
to decrease average interconnection length.

As shown with the A$^*$ algorithm, routing doesn't necessarily
have to take place on a strict grid of cells. Non-manhattan
graphs can be used in combination with the maze-routing
algorithms to more effectively plan routes and reduce
the total interconnect length.

