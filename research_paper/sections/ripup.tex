% !TeX root = ..\research_paper.tex

While partitioning certainly helps to solve the problem
of routing and reducing the amount of crossing and overall
interconnect length between nodes, the maze-routing
algorithms still suffer from some problems. Most notably,
they might suffer from significant congestion in high
traffic areas, and a manual technique for resolving
such conflicts is identifying which nets are problematic
and manually rerouting those. Or, as could be said,
``ripping'' and ``re-routing'' those nets. There have
been attempts to replicate the human behavaior of rerouting
with automated systems.

When maze routing fails (for example, the path has become
completely blocked by other nets), an example reroute
effect index for evaluating whether a net should be
rerouted can be shown with the following equation: $P(W)
= \text{min} \{N_A(W) - L_A(W), N_B(W) - L_B(W)\}$ (\cite{52212}).
This $P(W)$ is caluclated for each net $W$ in the region.

In this equation, $W$ represents a blocking net in the
area, and $N_A$ is the number of cells in that blocking
net that could be reached by the source cell of the
new path we are trying to reroute. Accordingly, $N_B$
is the number of cells in that blocking net that could
be reached by the destination cell of the new path.
$L_A$ is 1 if the both of the pins of the blocking net
are visible to the source cell of the new path, otherwise
0. $L_B$ is similarly defined. Utilizing this expression
for $P(W)$ we can determine whether attempting to reroute
the net is even worthwhile. If $P(W) >= 1$, then it
is worth attempting to reroute, otherwise no improvements
are possible for that net $W$.

Rip-up and rerouting is a technique that still has room
to improve, especially as newer techniques utilize the
analysis of multiple nets at a time, where this technique
currently shows some weakness. The implementation used here
is just an example of one rerouting technique that has
been implemented in some cases, and can be a basis for other
possibly more complex rip-up and rerouting techniques.
