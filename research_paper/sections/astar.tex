% !TeX root = ..\research_paper.tex

The A$^*$ algorithm (pronounced ``A-star'') is an algorithm
that essentially solves the same problem as Lee's algorithm,
though in a more time and memory-efficient way. It achieves
superior efficiency by estimating which paths are more
likely to find the solution earlier, resulting in fewer
cells checked.

The A$^*$ algorithm is implemented as follows. Imagining
a similar grid situation as in the example when considering
Lee's algorithm, the starting cell's neighbors will
all be considered, and their ``costs'' estimated with
a heuristic function. The heuristic function ($f(n)$)
follows the from $f(n) = g(n) + h(n)$, where $g(n)$
is the cost of getting to a certain cell from the starting
cell, and $h(n)$ is the estimated cost of getting from
that certain cell to the destination cell. You can think
of it essentially as an expansion of Lee's algorithm,
where Lee's algorithm considered only $g(h)$, neglecting
the cost of each neighbor's path towards the destination
cell.

Now, for each of those neighbors whose heuristic function
$f$ is lowest, you iterate and calculate the same for its
neighbors. Unlike Lee's algorithm, instead of searching
breadth-first in every direction from the starting cell,
the A$^*$ algorithm avoids evaluating or estimating the
cost of the cells which seem least likely to result in the
shortest path to the destination cell.

A deeper look into the A$^*$ algorithm shows that it may
not only be used for cells on a grid, but that it can
be used for any graph composed of nodes connected by
edges, where the cost of each edge is known. For example,
explored in the paper by \cite{4082128}, we see their
example of the implementation of the A$^*$ algorithm
for cities connected by roads. The algorithm may use
the calculated cost $g(n)$ (found as it traverses the graph
and sums the edge-costs) and the estimated $h(n)$ ``might
be the airline distance between city $n$ and the goal city.''
