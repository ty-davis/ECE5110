% !TeX root = ..\research_paper.tex

Lee's Algorithm was one of the first algorithms used
for grid-based automatic routing and was made in the
early 1960's. It was initially an endeavor to answer
the question, ``Is it possible to find procedures which
would enable a computer to solve efficiently path-connection
problems inherent in logical drawing, wiring diagramming,
and optimal route finding?'' (\cite{5219222}). In response
to that question, Lee says, ``The results are highly
encouraging.''

In today's standards the algorithm is rather simple,
but it was originally implemented on an IBM 704 computer,
and I believe that the idea is interesting enough to
be explored here. The premise of the algorithm is that
you have a set of cells in some space, and a way to
determine each cells' neighbors. The shortest path between
two cells, $c^i$ and $c^j$ can be determined in a breadth-first
approach by iteratively considering all of the neighbors
of $c^i$, and their neighbors, until you reach $c^j$.
To dive a little deeper into the implementation of the
algorithm, imagine a grid of cells of any size. Let
$c^i$ be on the left side, and you're trying to reach
$c^j$, which is placed on the right side. However, between
$c^i$ and $c^j$ is some obstacle. Some of the cells
between them are marked as an obstacle, representing
another wire connecting two other traces.

Starting at $c^i$ and marking it with the value $0$,
consider all of its neighbors (the four adjacent squares
in the grid) and mark them with the value $1$, assuming
that none of them are blocked. Repeat this process for
all of the neighbors, marking the subsequent neighbors
with increasingly higher values (1, 2, 3, \dots) until
you reach the destination cell $c^j$. Now, from $c^j$, simply
follow the cells marked with the lowest number
back to the original starting cell $c^i$. The number
that each cell is marked with can be interpreted as its distance
to the starting cell $c^i$.

Lee's algorithm is effective, but because of its breadth-first
nature, it becomes inefficient in memory and speed.
A number of other algorithms have been shown to improve
the performance of smallest path algorithms. Consider,
as well, that Lee's algorithm only considers paths on
a 2-dimensional plane, neglecting the use of metal-layers
and vias. Beyond this, Lee's algorithm may offer the
shortest path to whichever route is considered first,
when a more optimal layout for the overall design may
be achieved with an algorithm that considers which connections
to make first. Of course, that's why routing usually comes
as a later step, following placement and floorplanning.
