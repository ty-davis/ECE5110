% !TeX root = ..\research_paper.tex

One of the key things that I've learned in this class
is that a wealth of time working on a VLSI project can
be spent on the physical design and layout of a chip.
As with any task in engineering, efficiency and efficacy
in VLSI layout improves with practice, but with modern
transistor counts in computer chips approaching the
hundreds of billions, manually placing each transistor
becomes impossible. While many of the design structures
are repeated in large chips, such as the GPUs and CPUs
that boast such impressive transistor counts, hundreds
of thousands of decisions about cell location and layout
need to be made for the design of one chip. With each
decision affecting multiple important factors that should
be considered in VLSI design, using automated routing
and layout is necessary in the modern day.

Much of the motivation for the design of a chip comes
down to speed/performance, the power consumption, and
how easy the chip is to manufacture. Attempts to reduce
the total length of interconnects is a large priority
in automated routing, and decreasing unnecessary capacitance
improves the power-consumption, and therefore the heat
produced by the chip.

In this essay I will talk about some of the interesting
and important concepts that have emerged over the recent
decades that pertain to automatic layout and routing
techniques. Many of the techniques in VLSI design are
described as ``NP-complete'' or ``NP-hard problems''
(\cite{9357584}), where NP means ``non-deterministic
polynomial-time'' -- referencing the amount of time/compute
power required to verify that a solution is valid is
polynomial-time. Notably, the problems of cell placement,
floorplanning, routing, etc. would take so much time
to solve computationally that finding a perfect solution
is not viable. As such, many of the proposed techniques
rely on heuristics and approximations to find a near-complete
answer.
