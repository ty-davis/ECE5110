% !TeX root = ..\research_paper.tex

One of the first things that should be done in the physical
design process, considerably earlier than the maze routing
signals above, is partitioning the design. In the perspective
of VLSI design, imagine a large netlist showing the
connections between different logic cells in a graph
configuration. Choosing how to place which cells closer
to which can be a daunting task to do by hand, and a
good first step might be separate all of the cells into
smaller groups so that you can decide how to place them
in each group. Such a process is called ``partitioning'',
and the goal of an effective partitioning algorithm
is to keep the total number of connections between each
partition to a minimum.

The Kernighan-Lin algorithm in an effective algorithm
that relies on heuristics to produce good solutions
in a reasonable amount of time. The steps of the Kernighan-Lin
algorithm are once again straightforward. An example
follows that shows how to split a graph into two equal
(or about equal) subsets while minimizing the number/cost
of edge cuts between the two subsets.

First split the graph into subsets $A$ and $B$. With both
subsets separated, an estimate of the improvement that
would occur if you swapped each of the nodes to the
other subset is computed. Of course, calculating the
cost for each of the nodes for each of the subsets possible
in unrealistic beyond trivial examples, so an estimation
is necessary. In order to estimate the gain of swapping
a given node to the other subset, the following are
calculated. The ``internal cost'' ($I(v)$) is the sum
of edge weights to nodes in the same set. The ``external
cost'' ($E(v)$) is the sum of edge weights to nodes
in the other set. The D-value is then shown by $D(v)
= E(v) - I(v)$, this is how much the cost would improve
by moving node $v$ to the other side. A D-value is essentially
an estimate of how much better the cost would be if the
node $v$ were in the other subset.

After computing D-values for all of the nodes, find
which nodes are best to swap between the two subsets.
The equation used to do this is $g(a,b) = D(a) + D(b)
- 2 \cdot w(a,b)$, where $w(a,b)$ is the edge-cost between
the two nodes $a$ and $b$ if there is one. Subtracting
that term from the equation essentially disincentivizes
swapping two nodes if they are connected to each other, assuming
that there is a better swap available elsewhere.

This algorithm is effective because it attempts ``to
identify [the nodes to swap] from A and B, without considering
all possible choices'' (\cite{6771089}).
