\documentclass[12pt]{article}

%
%Margin - 1 inch on all sides
%
\usepackage{hyperref}
\usepackage[letterpaper]{geometry}
\usepackage{times}
\geometry{top=1.0in, bottom=1.0in, left=1.0in, right=1.0in}

%
%Doublespacing
%
\usepackage{setspace}
\singlespacing

%
%Rotating tables (e.g. sideways when too long)
%
\usepackage{rotating}

%
%Extras
%
\usepackage{siunitx}


%
%Fancy-header package to modify header/page numbering (insert last name)
%
\usepackage{fancyhdr}
\pagestyle{fancy}
\lhead{} 
\chead{} 
\rhead{Davis \thepage} 
\lfoot{} 
\cfoot{} 
\rfoot{} 
\renewcommand{\headrulewidth}{0pt} 
\renewcommand{\footrulewidth}{0pt} 
\setlength{\headheight}{14.5pt}
%To make sure we actually have header 0.5in away from top edge
%12pt is one-sixth of an inch. Subtract this from 0.5in to get headsep value
\setlength\headsep{0.333in}

%
%Bibtex and Bibliography
%
\usepackage[american]{babel}
\usepackage{csquotes}
\usepackage[style=mla,backend=biber]{biblatex}
\addbibresource{refs.bib}


%
%Begin document
%
\begin{document}
\begin{flushleft}

%%%%First page name, class, etc
Ty Davis\\
Dr. Justin Jackson\\
ECE 5110\\
April 20, 2025 \\


%%%%Title
\begin{center}
  Algorithms for Layout and Routing of VLSI Design
\end{center}



%%%%Changes paragraph indentation to 0.5in
\setlength{\parindent}{0.5in}
%%%%Begin body of paper here

% !TeX root = ..\research_paper.tex

One of the key things that I've learned in this class
is that a wealth of time working on a VLSI project can
be spent on the physical design and layout of a chip.
As with any task in engineering, efficiency and efficacy
in VLSI layout improves with practice, but with modern
transistor counts in computer chips approaching the
hundreds of billions, manually placing each transistor
becomes impossible. While many of the design structures
are repeated in large chips, such as the GPUs and CPUs
that boast such impressive transistor counts, hundreds
of thousands of decisions about cell location and layout
need to be made for the design of one chip. With each
decision affecting multiple important factors that should
be considered in VLSI design, using automated routing
and layout is necessary in the modern day.

Much of the motivation for the design of a chip comes
down to speed/performance, the power consumption, and
how easy the chip is to manufacture. Attempts to reduce
the total length of interconnects is a large priority
in automated routing, and decreasing unnecessary capacitance
improves the power-consumption, and therefore the heat
produced by the chip.

In this essay I will talk about some of the interesting
and important concepts that have emerged over the recent
decades that pertain to automatic layout and routing
techniques. Many of the techniques in VLSI design are
described as ``NP-complete'' or ``NP-hard problems''
(\cite{9357584}), where NP means ``non-deterministic
polynomial-time'' -- referencing the amount of time/compute
power required to verify that a solution is valid is
polynomial-time. Notably, the problems of cell placement,
floorplanning, routing, etc. would take so much time
to solve computationally that finding a perfect solution
is not viable. As such, many of the proposed techniques
rely on heuristics and approximations to find a near-complete
answer.


\subsection*{Maze Routing Algorithms}

\subsubsection*{Lee's Algorithm}
% !TeX root = ..\research_paper.tex

Lee's Algorithm was one of the first algorithms used
for grid-based automatic routing and was made in the
early 1960's. It was initially an endeavor to answer
the question, ``Is it possible to find procedures which
would enable a computer to solve efficiently path-connection
problems inherent in logical drawing, wiring diagramming,
and optimal route finding?'' (\cite{5219222}). In response
to that question, Lee says, ``The results are highly
encouraging.''

In today's standards the algorithm is rather simple,
but it was originally implemented on an IBM 704 computer,
and I believe that the idea is interesting enough to
be explored here. The premise of the algorithm is that
you have a set of cells in some space, and a way to
determine each cells' neighbors. The shortest path between
two cells, $c^i$ and $c^j$ can be determined in a breadth-first
approach by iteratively considering all of the neighbors
of $c^i$, and their neighbors, until you reach $c^j$.
To dive a little deeper into the implementation of the
algorithm, imagine a grid of cells of any size. Let
$c^i$ be on the left side, and you're trying to reach
$c^j$, which is placed on the right side. However, between
$c^i$ and $c^j$ is some obstacle. Some of the cells
between them are marked as an obstacle, representing
another wire connecting two other traces.

Starting at $c^i$ and marking it with the value $0$,
consider all of its neighbors (the four adjacent squares
in the grid) and mark them with the value $1$, assuming
that none of them are blocked. Repeat this process for
all of the neighbors, marking the subsequent neighbors
with increasingly higher values (1, 2, 3, \dots) until
you reach the destination cell $c^j$. Now, from $c^j$, simply
follow the cells marked with the lowest number
back to the original starting cell $c^i$. The number
that each cell is marked with can be interpreted as its distance
to the starting cell $c^i$.

Lee's algorithm is effective, but because of its breadth-first
nature, it becomes inefficient in memory and speed.
A number of other algorithms have been shown to improve
the performance of smallest path algorithms. Consider,
as well, that Lee's algorithm only considers paths on
a 2-dimensional plane, neglecting the use of metal-layers
and vias. Beyond this, Lee's algorithm may offer the
shortest path to whichever route is considered first,
when a more optimal layout for the overall design may
be achieved with an algorithm that considers which connections
to make first. Of course, that's why routing usually comes
as a later step, following placement and floorplanning.


\subsubsection*{A$^*$ Algorithm}
% !TeX root = ..\research_paper.tex

The A$^*$ algorithm (pronounced ``A-star'') is an algorithm
that essentially solves the same problem as Lee's algorithm,
though in a more time and memory-efficient way. It achieves
superior efficiency by estimating which paths are more
likely to find the solution earlier, resulting in fewer
cells checked.

The A$^*$ algorithm is implemented as follows. Imagining
a similar grid situation as in the example when considering
Lee's algorithm, the starting cell's neighbors will
all be considered, and their ``costs'' estimated with
a heuristic function. The heuristic function ($f(n)$)
follows the from $f(n) = g(n) + h(n)$, where $g(n)$
is the cost of getting to a certain cell from the starting
cell, and $h(n)$ is the estimated cost of getting from
that certain cell to the destination cell. You can think
of it essentially as an expansion of Lee's algorithm,
where Lee's algorithm considered only $g(h)$, neglecting
the cost of each neighbor's path towards the destination
cell.

Now, for each of those neighbors whose heuristic function
$f$ is lowest, you iterate and calculate the same for its
neighbors. Unlike Lee's algorithm, instead of searching
breadth-first in every direction from the starting cell,
the A$^*$ algorithm avoids evaluating or estimating the
cost of the cells which seem least likely to result in the
shortest path to the destination cell.

A deeper look into the A$^*$ algorithm shows that it may
not only be used for cells on a grid, but that it can
be used for any graph composed of nodes connected by
edges, where the cost of each edge is known. For example,
explored in the paper by \cite{4082128}, we see their
example of the implementation of the A$^*$ algorithm
for cities connected by roads. The algorithm may use
the calculated cost $g(n)$ (found as it traverses the graph
and sums the edge-costs) and the estimated $h(n)$ ``might
be the airline distance between city $n$ and the goal city.''


\subsubsection*{Manhattan vs Non-Manhattan Routing}
% !TeX root = ..\research_paper.tex

As of the early 2000's, ``most routing algorithms use[d]
a Manhattan geometry with horizontal and vertical traces''
(\cite{1360580}). Such designs prevailed because of
their simplicity, but the increase in wire length (27\%
longer on average than the Euclidean distance) of the
interconnections resulted in significantly slower circuits.
The concept of ``non-Manhattan'' routing, therefore,
is any routing that doesn't strictly adhere to a grid.

Non-Manhattan layout designs can be applied not only
to VLSI design projects, but also PCB and other circuit
layouts. The study cited above showed the use of two
different Non-Manhattan layout algorithms that routed
wires on horizontal, vertical, and \ang{45} diagonal
lines, but other non-Manhattan layout paradigms exist
which operate on other bases. While the results from
that study can only be applied to the specific layout
algorithms tested, it is safe to say that Non-Manhattan
layouts should always be a consideration when trying
to decrease average interconnection length.

As shown with the A$^*$ algorithm, routing doesn't necessarily
have to take place on a strict grid of cells. Non-manhattan
graphs can be used in combination with the maze-routing
algorithms to more effectively plan routes and reduce
the total interconnect length.



\subsubsection*{Miscellaneous Techniques}
% !TeX root = ..\research_paper.tex

There are other widely used techniques in 
routing and planning that can be utilized in VLSI design
that I felt deserved to be mentioned here.

River-routing is the concept of laying out the connections
between cells such that the paths don't have to cross, allowing
the lines to look like a parallel streams, or the illusion of a river.

X-routing and Y-routing is a similar technique where
you use two different metal layers in different directions.
For instance, a metal1 layer may be used East-West (on the x-axis)
while a metal2 layer is used from North-South (on the y-axis).


\subsection*{Placement and Planning}

The planning algorithms we've considered so far are
limited in their direct application to VLSI routing
and planning. Maze routing algorithms consider only
how to connect two points within a 2-dimensional space,
or a space that can otherwise be demonstrated in the
form of a graph. Such algorithms don't consider the
use of vias, other metal layers, nor do they consider
more than one path at a time.

There are more techniques within graph theory that can
be used to improve and optimize the layout of a design,
and they may have a more direct impact on the overall
VLSI design that is ultimately achieved. 

\subsubsection*{Partitioning with the Kernighan-Lin Algorithm}
% !TeX root = ..\research_paper.tex

One of the first things that should be done in the physical
design process, considerably earlier than the maze routing
signals above, is partitioning the design. In the perspective
of VLSI design, imagine a large netlist showing the
connections between different logic cells in a graph
configuration. Choosing how to place which cells closer
to which can be a daunting task to do by hand, and a
good first step might be separate all of the cells into
smaller groups so that you can decide how to place them
in each group. Such a process is called ``partitioning'',
and the goal of an effective partitioning algorithm
is to keep the total number of connections between each
partition to a minimum.

The Kernighan-Lin algorithm in an effective algorithm
that relies on heuristics to produce good solutions
in a reasonable amount of time. The steps of the Kernighan-Lin
algorithm are once again straightforward. An example
follows that shows how to split a graph into two equal
(or about equal) subsets while minimizing the number/cost
of edge cuts between the two subsets.

First split the graph into subsets $A$ and $B$. With both
subsets separated, an estimate of the improvement that
would occur if you swapped each of the nodes to the
other subset is computed. Of course, calculating the
cost for each of the nodes for each of the subsets possible
in unrealistic beyond trivial examples, so an estimation
is necessary. In order to estimate the gain of swapping
a given node to the other subset, the following are
calculated. The ``internal cost'' ($I(v)$) is the sum
of edge weights to nodes in the same set. The ``external
cost'' ($E(v)$) is the sum of edge weights to nodes
in the other set. The D-value is then shown by $D(v)
= E(v) - I(v)$, this is how much the cost would improve
by moving node $v$ to the other side. A D-value is essentially
an estimate of how much better the cost would be if the
node $v$ were in the other subset.

After computing D-values for all of the nodes, find
which nodes are best to swap between the two subsets.
The equation used to do this is $g(a,b) = D(a) + D(b)
- 2 \cdot w(a,b)$, where $w(a,b)$ is the edge-cost between
the two nodes $a$ and $b$ if there is one. Subtracting
that term from the equation essentially disincentivizes
swapping two nodes if they are connected to each other, assuming
that there is a better swap available elsewhere.

This algorithm is effective because it attempts ``to
identify [the nodes to swap] from A and B, without considering
all possible choices'' (\cite{6771089}).


\subsubsection*{Rip-Up and Reroute}
% !TeX root = ..\research_paper.tex

Ripup


\subsubsection*{Simulated Annealing}
% !TeX root = ..\research_paper.tex

Annealing here!


\subsection*{Machine Learning}

Modern-day improvements in layout and routing are being
explored through the use of machine learning. Machine
learning is a massively expanding branch of computer science
that has proven to tackle large-scale problems like these
with lots of success. Many forms of machine learning
exist that can be used for problem solving, each with their
pros and cons.

\subsubsection*{Reinforcement Learning}
% !TeX root = ..\research_paper.tex

Reinforcement learning (RL, or DRL for deep reinforcement
learning) is a form of machine learning where an agent
will perform a task (such as floorplanning or routing)
and then the solution will be evaluated and given either
a punishment or a reward. The agent then ``learns''
from that reward. The solutions that are given rewards
will be replicated and iterated on, while those that
are not will be rejected in future iterations. Hundreds
of thousands of examples will be used in order for the
neural network to effectively learn and perform in a
way that fits design needs. A group from the Google
Chip Implementation showed their results and methods
in a 2020 paper. They mentioned that the reward system
for their DRL was based on ``wirelength, because it
is not only much cheaper to evaluate, but also correlates
with power and performance (timing)'' (\cite{DBLP:journals/corr/abs-2004-10746}).

The paper outlined performance improvements that resulted
from using RL instead of simulated annealing processes. On
average, wirelength and congestion decreased, meaning that
overall power consumption and performance would be improved.


\subsubsection*{Conclusion}
% !TeX root = ..\research_paper.tex

This is the conclusion



\newpage
% \nocite{*}

\printbibliography

\end{flushleft}
\end{document}
